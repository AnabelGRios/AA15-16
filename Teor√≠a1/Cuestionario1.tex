\documentclass[12pt]{article}

\usepackage{lmodern}
\usepackage[T1]{fontenc}
\usepackage[spanish,activeacute]{babel}
\usepackage[utf8]{inputenc}
\usepackage{mathtools}
\usepackage{enumerate}
\usepackage{amsthm}
\usepackage{amssymb}
\usepackage[hidelinks]{hyperref}

\title{Aprendizaje Automático: Cuestionario 1}
\author{Anabel G\'omez R\'ios}

\theoremstyle{definition}

\begin{document}
\maketitle

\newtheorem{pregunta}{Pregunta}

\begin{pregunta}
Identificar, para cada una de las siguiente tareas, qué tipo de aprendizaje automático es el adecuado (supervisado, no supervisado, por refuerzo) y los datos de aprendizaje que deberíamos usar. Si una tarea se ajusta a más de un tipo, explicar cómo y describir los datos para cada tipo.
\begin{enumerate}
\item[a)] Categorizar un grupo de animales vertebrados en pájaros, mamíferos, reptiles, aves y anfibios.
\item[b)] Clasificación automática de cartas por distrito postal.
\item[c)] Decidir si un determinado índice del mercado de valores subirá o bajará dentro de un periodo de tiempo fijado.
\end{enumerate}
\textit{ }\\


\end{pregunta}

\begin{pregunta}
¿Cuáles de los siguientes problemas son más adecuados para una aproximación por aprendizaje y cuáles más adecuados para una aproximación por diseño? Justificar la decisión.
\begin{enumerate}
\item[a)] Determinar el ciclo óptimo para las luces de los semáforos en un cruce con mucho tráfico.
\item[b)] Determinar los ingresos medios de una persona a partir de sus datos de nivel de educación, edad, experiencia y estatus social.
\item[c)] Determinar si se debe aplicar una campaña de vacunación contra una enfermedad.
\end{enumerate}
\textit{ }\\

\end{pregunta}

\begin{pregunta}
Construir un problema de \textit{aprendizaje desde datos} para un problema de selección de fruta en una explotación agraria (ver transparencias de clase). Identificar y describir cada uno de sus elementos formales. Justificar las decisiones.\\


\end{pregunta}

\begin{pregunta}
Suponga un modelo PLA y un dato $x(t)$ mal clasificado respecto de dicho modelo. Probar que la regla de adaptación de pesos del PLA es un movimiento en la dirección correcta para clasificar bien $x(t)$.\\

\end{pregunta}

\begin{pregunta}
Considere el enunciado del ejercicio 2 de la sección FACTIBILIDAD DEL APRENDIZAJE de la relación de apoyo.
\begin{enumerate}
\item[a)] Si $p=0,9$, ¿cuál es la probabilidad de que $S$ produzca una hipótesis mejor que $C$?
\item[b)] ¿Existe un valor de $p$ para el cual es más probable que $C$ produzca una hipótesis mejor que $S$?
\end{enumerate}
\textit{ }\\


\end{pregunta}

\begin{pregunta}
La desigualdad de Hoeffding modificada nos da una forma de caracterizar el error de generalización con una cota probabilística
\begin{equation}
P[|E_{out}(g) - E_{in}(g)| > \epsilon] \leqslant 2Me^{-2N^2 \epsilon}
\end{equation}
para cualquier $\epsilon > 0$. Si fijamos $\epsilon=0,05$ y queremos que la cota probabilística $2Me^{-2N^2 \epsilon}$ sea como máximo $0,03$, ¿cuál será el valor más pequeño de $N$ que verifique estas condiciones si $M=1$? Repetir para $M=10$ y para $M=100$.\\

\end{pregunta}

\begin{pregunta}
Consideremos el modelo de aprendizaje "M-intervalos" \textit{ }donde $h: \mathbb{R} \rightarrow {-1, +1}$ y $h(x)=+1$ si el punto está dentro de cualquiera de $m$ intervalos arbitrariamente elegidos y $-1$ en otro caso. ¿Cuál es el más pequeño punto de ruptura para este conjunto de hipótesis?\\


\end{pregunta}

\begin{pregunta}
Suponga un conjunto de $k^*$ puntos $x_1,x_2,...,x_{k^*}$ sobre los cuales la clase $H$ implementa $<2^{k^*}$ dicotomías. ¿Cuáles de las siguientes afirmaciones son correctas?
\begin{enumerate}
\item[a)] $k^*$ es un punto de ruptura
\item[b)] $k^*$ no es un punto de ruptura
\item[c)] todos los puntos de ruptura son estrictamente mayores que $k^*$
\item[d)] todos los puntos de ruptura son menores o iguales a $k^*$
\item[e)] no conocemos nada acerca del punto de ruptura
\end{enumerate}
\textit{ }\\

\end{pregunta}

\begin{pregunta}
Para todo conjunto de $k^*$ puntos, $H$ implementa $<2^{k^*}$ dicotomías. ¿Cuáles de las siguientes afirmaciones son correctas?
\begin{enumerate}
\item[a)] $k^*$ es un punto de ruptura
\item[b)] $k^*$ no es un punto de ruptura
\item[c)] todos los $k \geqslant k^*$ son puntos de ruptura
\item[d)] todos los $k < k^*$ son puntos de ruptura
\item[e)] no conocemos nada acerca del punto de ruptura
\end{enumerate}
\textit{ }\\

\end{pregunta}

\begin{pregunta}
Si queremos mostrar que $k^*$ es un punto de ruptura, ¿cuáles de las siguientes afirmaciones nos servirían para ello?:
\begin{enumerate}
\item[a)] Mostrar que existe un conjunto de $k^*$ puntos $x_1,...x_{k^*}$ que $H$ puede separar ("shatter").
\item[b)] Mostrar que $H$ puede separar cualquier conjunto de $k^*$ puntos.
\item[c)] Mostrar un conjunto de $k^*$ puntos $x_1,...,x_{k^*}$ que $H$ no puede separar.
\item[d)] Mostrar que $H$ no puede separar ningún conjutno de $k^*$ puntos.
\item[e)] Mostrar que $m_H(k)=2^{k^*}$
\end{enumerate}
\textit{ }\\

\end{pregunta}

\begin{pregunta}
Para un conjunto $H$ con $d_{VC}=10$, ¿qué tamaño muestral se necesita (según la cota de generalización) para tener un $95\%$ de confianza de que el error de generalización sea como mucho $0,05$?\\


\end{pregunta}

\begin{pregunta}
Consideremos un escenario de aprendizaje simple. Supongamos que la dimensión de entrada es uno. Supongamos que la variable de entrada $x$ está uniformemente distribuida en el intervalo $[-1,1]$ y el conjunto de datos consiste en 2 puntos ${x_1,x_2}$ y que la función objetivo es $f(x)=x^2$. Por tanto el conjunto de datos completo es $D={(x_1,x_1^2), (x_2,x_2^2)}$. El algoritmo de aprendizaje devuelve la línea que ajusta estos dos puntos como $g$ (i.e. $H$ consiste en funciones de la forma $h(x)=ax+b$).
\begin{enumerate}
\item[a)] Dar una expresión analítica para la función promedio $\overline{g}(x)$.
\item[b)] Calcular analíticamente los valores de $E_{out}$, \textbf{bias} y \textbf{var}.
\end{enumerate}
\textit{ }\\


\end{pregunta}
\end{document}